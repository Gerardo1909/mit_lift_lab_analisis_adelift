El presente informe constituye una primera aproximación descriptiva sobre las percepciones y expectativas de nanostores en Argentina, en un contexto de alta volatilidad macroeconómica y shocks coyunturales. A partir de este punto, se identifican varias líneas de trabajo para profundizar el análisis:

\begin{itemize}
    \item \textbf{Ampliación de la base de datos:} Se planea continuar la recolección de encuestas, tanto incorporando nuevos micro y pequeños comercios como reencuestando a los participantes originales. Esto permitirá analizar la evolución de percepciones y expectativas a lo largo del tiempo, especialmente para estudiar el impacto persistente de shocks coyunturales y la presencia de dinámicas estacionales.
    \item \textbf{Análisis de lenguaje natural:} Se propone profundizar el análisis de las respuestas abiertas mediante técnicas de procesamiento de lenguaje natural (NLP). Esto permitirá identificar patrones temáticos, cambios en el sentimiento y la aparición de nuevas preocupaciones o expectativas a medida que evoluciona el contexto económico.
    \item \textbf{Modelado analítico:} A medida que la base de datos crezca, se evaluará la implementación de modelos estadísticos o de aprendizaje automático para predecir expectativas o identificar factores que explican la alineación (o falta de alineación) entre percepciones recientes y expectativas futuras. El enfoque inicial estará en modelos descriptivos y exploratorios, priorizando la interpretabilidad.
    \item \textbf{Profundización en estacionalidad y shocks:} Con datos longitudinales, será posible analizar cómo los negocios ajustan sus expectativas tras eventos específicos (por ejemplo, shocks macroeconómicos, cambios de política, fechas comerciales relevantes) y si existen diferencias sistemáticas según tipo de negocio o antigüedad.
\end{itemize}

Estos pasos permitirán no solo enriquecer el diagnóstico actual, sino también sentar las bases para recomendaciones más precisas y adaptadas a la realidad de los nanostores en contextos de alta incertidumbre.