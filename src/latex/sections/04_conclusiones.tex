En conjunto, los resultados describen un escenario típico del microcomercio de cercanía en contextos volátiles: el comerciante convive con señales contradictorias (estacionalidad comercial vs. incertidumbre coyuntural) y traduce esa tensión en decisiones prácticas. En las respuestas abiertas, aparecen menciones a eventos y al clima económico; en las expectativas, se observa una apuesta de corto plazo a la estacionalidad.

Como aporte principal, el informe propone operacionalizar esa tensión a través de una métrica simple de consistencia entre expectativas y percepciones. En iteraciones futuras, este enfoque podría ampliarse con:

\begin{itemize}
  \item Un análisis cuantitativo formal (por ejemplo, modelos simples por rubro y antigüedad).
  \item Un análisis de texto más robusto (por ejemplo, clasificación temática o sentimiento, si se incorpora un método validado para español y se controlan sesgos).
\end{itemize}
