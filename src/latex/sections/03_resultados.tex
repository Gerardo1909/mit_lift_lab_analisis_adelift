\subsection{Señales estacionales en el relato de los comerciantes (análisis de texto)}

Las preguntas abiertas permiten observar cómo los comerciantes explican ``por qué'' subieron o bajaron las ventas. En un primer nivel, suelen aparecer términos generales (por ejemplo, referencias a ``gente/clientes'' o a la ``situación económica''), coherentes con el contexto macro que enfrentan los hogares \parencite{indec_ipc}.

En octubre, también emergen términos asociados a dinámicas estacionales (por ejemplo, menciones a ``madre'', ``temporada'' o ``fin de mes''), que funcionan como señales narrativas de picos de consumo o cambios transitorios en la demanda. En paralelo, aparecen referencias a ``elecciones'' y a ``precios'', que suelen asociarse en el discurso de los encuestados a incertidumbre, cautela o ajustes preventivos.

\begin{figure}[H]
  \centering
  \includegraphics[width=0.95\linewidth]{figura_1_nube_palabras_ventas.png}
  \caption{Términos más frecuentes en las razones declaradas de cambio en ventas.}
  \caption*{\small\textit{Fuente:} Elaboración propia con datos del MIT LIFT Lab (octubre de 2025).}
\end{figure}

Importante: este análisis corresponde a frecuencia de términos (no a un puntaje cuantitativo de sentimiento). Aun así, es útil para identificar qué temas dominan el relato y qué eventos o preocupaciones aparecen espontáneamente.

\subsection{Expectativas de ventas a tres meses en contexto de volatilidad}

Las expectativas de ventas a 3 meses muestran cómo los dueños proyectan su desempeño en un horizonte corto/medio. Un resultado relevante es que, incluso cuando el contexto se percibe incierto, suele persistir una fracción importante de respuestas optimistas (por ejemplo, ``mayores'') que se asocia a la inercia estacional hacia fin de año.

Esta coexistencia---preocupación en el relato cualitativo y expectativa relativamente optimista---es consistente con la idea de que los nanostores aprenden a operar bajo volatilidad, apoyándose en patrones culturales de consumo (picos estacionales) como referencia práctica para planificar \parencite{mitlift}.

\begin{figure}[H]
  \centering
  \includegraphics[width=0.80\linewidth]{figura_2_expectativa_ventas_3m.png}
  \caption{Distribución de la expectativa de ventas en un horizonte de 3 meses.}
  \caption*{\small\textit{Fuente:} Elaboración propia con datos del MIT LIFT Lab (octubre de 2025).}
\end{figure}

\subsection{Consistencia entre expectativa y desempeño reciente (brecha expectativa--resultado)}

Para conectar expectativas con resultados recientes, se construye un indicador simple de consistencia comparando:

\begin{itemize}
  \item La expectativa de ventas a 3 meses (declarada al momento de la encuesta), y
  \item La percepción del desempeño reciente (ventas respecto del mes anterior).
\end{itemize}

La idea es estimar, por tipo de negocio y antigüedad (en grupos), qué tan alineadas están las expectativas con el desempeño reportado. Este indicador no pretende medir desempeño ``real'' (no hay auditoría de ventas), pero sí comparar consistencia interna entre percepción y proyección.

\begin{figure}[H]
  \centering
  \includegraphics[width=0.95\linewidth]{figura_3_consistencia_expectativa_vs_ventas.png}
  \caption{Porcentaje de coincidencia entre expectativa a 3 meses y desempeño reciente, por tipo y antigüedad del negocio.}
  \caption*{\small\textit{Fuente:} Elaboración propia con datos del MIT LIFT Lab (octubre de 2025).}
\end{figure}

\noindent\textbf{Cómo leer la figura.} Un porcentaje alto indica que, dentro de ese grupo, la expectativa a 3 meses suele estar \emph{alineada en dirección} con el desempeño reciente reportado (por ejemplo, ``mejor'' con ``mejor''). Un porcentaje bajo sugiere que las expectativas \emph{no} siguen de cerca el último mes.

\noindent\textbf{Vínculo con estacionalidad y shocks.} En un contexto de estacionalidad, una menor alineación puede reflejar que los comerciantes proyectan el próximo trimestre en función del calendario comercial (picos transitorios) más que del mes inmediato anterior. En un contexto de shocks e incertidumbre, una menor alineación puede reflejar cautela o volatilidad en las proyecciones; alternativamente, un shock que afecte de manera generalizada podría aumentar la alineación si el deterioro/mejora reciente se extrapola.
