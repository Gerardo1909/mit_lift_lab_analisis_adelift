Los nanostores (almacenes, kioscos, verdulerías y comercios de cercanía similares) son un componente clave de los sistemas de consumo barrial y de la economía cotidiana. Sin embargo, su escala reducida y su dependencia del flujo de caja diario los vuelve especialmente sensibles a cambios repentinos en la demanda, expectativas y precios. En contextos de inflación elevada, la gestión de inventario, la fijación de precios y la planificación de compras se vuelven decisiones de supervivencia \parencite{indec_ipc}.

En ese marco, octubre de 2025 resulta una ventana relevante para observar la superposición de señales estacionales---asociadas a picos de consumo por fechas comerciales, como el Día de la Madre---con un entorno de incertidumbre coyuntural (por ejemplo, cambios en expectativas macro y el clima político, como un proceso electoral). Este trabajo se concentra en cómo los comerciantes describen y anticipan el desempeño de su negocio: qué reportan sobre sus ventas recientes, qué esperan hacia adelante y qué razones mencionan en respuestas abiertas.

\newpage